\documentclass{article}
\usepackage[english]{babel}
\usepackage[utf8]{inputenc}
\usepackage{fancyhdr}
\usepackage{graphicx}
\usepackage{geometry}
\usepackage{float}
\usepackage{hyperref}
\usepackage{amsmath}

 \newgeometry{vmargin={15mm}, hmargin={12mm,17mm}}

\pagestyle{fancy}
\fancyhf{}
\rhead{COMP20003 S2 2018\\Jack Zezula (808867)}
\lhead{Assignment 2 - B15 Puzzle Solver}

\begin{document}

\section{Program Overview}
  This B-15 puzzle solving program utilised two node instances: an initial node
  and a current-state node. Upon each increase of the threshold, the state node
  was reset to the initial node.\\
  Utilising only two node instances in memory provided a relatively low memory
  usage by the program (approximately 64 kilobytes). \\
  Prior to testing the provided puzzle configurations, the program relied on
  exploring all possible actions from each node. However, this implementation
  was observed to take over 10 minutes for even the medium-complexity puzzle of
  2.puzzle:
  $$ \begin{bmatrix}
    13 & 5 & 4 &10 \\
     9 &12 & 8 &14 \\
     2 & 3 & 7 & 1 \\
     0 &15 &11 & 6
  \end{bmatrix} $$
  Due to processing limitations, it seemed that testing the more complex
  puzzles was out of the question. \\
  To fix this issue, the logic was corrected by preventing the program from
  exploring nodes which it had already explored for a given threshold. This
  sped up the program by many orders of magnitude, with 2.puzzle taking only
   $\approx 0.3$ seconds.

\section{Heuristic Improvements}
  The algorithm the B-15 puzzle solution was observed with two different heuristic functions:
  \begin{enumerate}
    \item Manhattan Distance
    \item Manhattan Distance with Last Moves Optimisation
  \end{enumerate}
  Overall, the heuristic optimisation outperformed the basic Manhattan
  heuristic when it came to time performance in the more complex examples
  (puzzles 1, 3, 88). The optimisation also consistently decreased the number
  of expanded nodes required to obtain a solution. However, this came at the
  cost of roughly halving the expansion speed ($nodes/sec$) across all test
  instances. \\
  Although this implementation did not consistently improve performance across
  all test cases, its decrease in the execution-time of puzzle 88 (arguably
  the most complex puzzle, expecting roughly $6003M$ expanded nodes) and its
  heavy reduction in the number of expanded nodes ($1090M$) suggests that it
  is an efficient optimisation.
  \\This optimisation was informed by the paper
  \href{http://www.aaai.org/Papers/AAAI/1996/AAAI96-178.pdf}{'Finding Optimal Solutions to the Twenty-Four Puzzle' by Richard E. Korf and Larry A. Taylor (1996)}.

\section{Results}
\subsection{No Optimisation}
  \begin{table}[H]
    \resizebox{\columnwidth}{!}{
    \begin{tabular}{l|l|l|l|l|l|l}
    ID & h($s_0$) & Thresholds                          & Optimal Moves & Expanded Nodes & Expanded/sec & Time (secs) \\
    1  & 41    & 41 43 45 47 49 51 53 55 57          & 57            & 206,131,932    & 16,813,273   & 12.26       \\
    2  & 43    & 43 45 47 49 51 53 55                & 55            & 4,636,976      & 16,416,979   & 0.28        \\
    3  & 41    & 41 43 45 47 49 51 53 55 57 59       & 59            & 229,658,354    & 15,334,415   & 14.98       \\
    4  & 42    & 42 44 46 48 50 52 54 56             & 56            & 41,689,053     & 16,457,564   & 2.53        \\
    14 & 41    & 41 43 45 47 49 51 53 55 57 59 61    & 61            & 1,090,399,330    & 17,037,135   & 64.00       \\
    88 & 43    & 43 45 47 49 51 53 55 57 59 61 63 65 & 65            & 6,610,107,370  & 16,972,586   & 389.46
    \end{tabular}
    }
  \end{table}
\subsection{Last Moves Optimisation}
  \begin{table}[H]
    \resizebox{\columnwidth}{!}{
      \begin{tabular}{l|l|l|l|l|l|l}
      ID & h($s_0$) & Thresholds                       & Optimal Moves & Expanded Nodes & Expanded/sec & Time (secs) \\
      1  & 41       & 41 43 45 47 49 51 53 55 57       & 57            & 77,138,658     & 8,313,906    & 9.28        \\
      2  & 43       & 43 47 49 51 53 55                & 55            & 2,937,285      & 8,724,237    & 0.34        \\
      3  & 41       & 41 45 47 49 51 53 55 57 59       & 59            & 95,347,992     & 8,742,682    & 10.91       \\
      4  & 42       & 42 44 46 48 50 52 54 56          & 56            & 33,152,534     & 9,238,433    & 3.59        \\
      14 & 41       & 41 43 45 47 49 51 53 55 57 59 61 & 61            & 1,034,575,816  & 9,437,904    & 109.62      \\
      88 & 43       & 43 47 49 51 53 55 57 59 61 63 65 & 65            & 1,879,404,554  & 8,404,575    & 223.62
      \end{tabular}
    }
  \end{table}

\end{document}
